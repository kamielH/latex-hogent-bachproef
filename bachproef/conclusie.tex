%%=============================================================================
%% Conclusie
%%=============================================================================

\chapter{Conclusie}%
\label{ch:conclusie}

% TODO: Trek een duidelijke conclusie, in de vorm van een antwoord op de
% onderzoeksvra(a)g(en). Wat was jouw bijdrage aan het onderzoeksdomein en
% hoe biedt dit meerwaarde aan het vakgebied/doelgroep? 
% Reflecteer kritisch over het resultaat. In Engelse teksten wordt deze sectie
% ``Discussion'' genoemd. Had je deze uitkomst verwacht? Zijn er zaken die nog
% niet duidelijk zijn?
% Heeft het onderzoek geleid tot nieuwe vragen die uitnodigen tot verder 
%onderzoek?

I've looked into this a bit, and here are my comments:
1.	The MQ-7 module is so poorly documented that it is not a great choice to work with
2.	There are other (qualitative) sensors (such as the mics5524) that seem simpler to work with, and don't appear to have any clear disadvantage vs. the MQ-7.
3.	If I were to take the time to calibrate the MQ-7 fully, it would only really be of use to me. Unless I mapped out the entire RH, temperature, CO concentration, and interfering gases profile, it is unlikely my results would be directly applicable to anyone else. Additionally, even if I did so, anyone looking to make quantitative measurements would (hopefully) repeat some of my results under their exact conditions with their own sensor to ensure they got the same data. By the time this was complete, their own data would be more useful for them than what I could provide.
4.	In case it is of help to anyone, I am posting the circuit that was suggested by Leo. I have built this circuit, and it worked exactly as expected (thanks Leo). While I did physically build the circuit, for this group, I am also posting the results of the circuit simulation in case anyone finds it useful.




