%%=============================================================================
%% Conclusie
%%=============================================================================

\chapter{Conclusie}%
\label{ch:conclusie}

% TODO: Trek een duidelijke conclusie, in de vorm van een antwoord op de
% onderzoeksvra(a)g(en). Wat was jouw bijdrage aan het onderzoeksdomein en
% hoe biedt dit meerwaarde aan het vakgebied/doelgroep? 
% Reflecteer kritisch over het resultaat. In Engelse teksten wordt deze sectie
% ``Discussion'' genoemd. Had je deze uitkomst verwacht? Zijn er zaken die nog
% niet duidelijk zijn?
% Heeft het onderzoek geleid tot nieuwe vragen die uitnodigen tot verder 
%onderzoek?

%centrale onderzoeksvraag: Hoe geschikt zijn goedkope sensoren om gassen in stalomgevingen te meten?
%deelonderzoeksvragen:  - Zijn deze gassensoren geschikt om de luchtsamenstelling uit te lezen?
%                       - Zijn deze gassensoren geschikt om stalgassen te meten?
%                       - Welke voordelen heeft een warmte-koelcyclus in een sensor?


In deze studie werd onderzocht hoe geschikt goedkope sensoren zijn om gassen in stalomgeving te meten. Dit heeft verschillende inzichten opgeleverd.

Allereerst is er vastgesteld dat de MQ-sensoren niet optimaal zijn voor het nauwkeurig meten van de luchtsamenstelling. De MQ-sensoren zijn zeer instabiel en niet nauwkeurig. Ook is de beschikbare informatie in de datasheets beperkt, waardoor het kalibratieproces moeilijk kan verlopen.

In sectie~\ref{sec:nauwkeurigheid} werden er 2 veelvoorkomende stalgassen getest.
In subsectie~\ref{subsec:nh3} werden de 3 sensoren onderworpen aan NH\textsubscript{3}. Hier werd duidelijk dat de MQ-sensoren niet heel stabiel waren. Er werd vastgesteld dat de MQ-7 en -135 sensoren gevoelig zijn voor ammoniak, maar doordat ammoniak niet wordt vermeld in de datasheet kon de ppm niet worden berekend en vergeleken.
In subsectie~\ref{subsec:co2} werd het tweede stalgas CO\textsubscript{2} getest. Hier werd duidelijk dat de MQ-135 sensor, na een correcte berekening van de R\textsubscript{0}-waarde, wel een ongeveer de juiste waarde ppm CO\textsubscript{2} kon inschatten, hoewel dit niet zeer stabiel was. Verder waren de MQ-4 en -7 sensoren ook gevoelig, maar in een mindere mate dan de MQ-135.

De volledige luchtsamenstelling is slechts een berekende schatting op basis van 1 waarde, waardoor deze met een flinke korrel zout moet worden genomen.

Desondanks de instabiliteit kunnen deze sensoren wel nog steeds detecteren of er gas aanwezig is. Aangezien dit een zeer goedkope gassensor is die een zeer snelle respons geeft kan deze dus wel bruikbaar zijn als goedkope monitor. Zo zou het via een alarmsignaal kunnen alarmeren wanneer er een te slechte luchtkwaliteit is.

Hiernaast werd ook de toepassing van een warmte-koelcyclus uitgetest.
De toepassing om verschillende gassen te onderscheiden heeft potentieel voor verder onderzoek. Aangezien elk gas een unieke respons vertoont in de grafieken die ontstaan bij deze warmte-koelcyclus. Dit zou mogelijks kunnen worden verbeterd en verfijnd om op deze manier een onderscheid te kunnen maken tussen verschillende gassen.

Het nadeel van deze warmte-koelcyclus is dat er slechts om de twee en een halve minuut een waarde kan worden uitgelezen. Ook zorgt deze warmte-koelcyclus voor een complexere opstelling en meer energieverbruik dan normaal.



In conclusie, hoewel de MQ-sensoren beperkingen hebben wat betreft nauwkeurigheid en stabiliteit, bieden ze toch mogelijkheden om op een goedkope manier de luchtkwaliteit te monitoren en zo potentiële gevaren te identificeren. Verdere studies zouden zich kunnen richten op het verbeteren van de nauwkeurigheid van deze sensoren. Alsook het verkennen van nieuwe methoden voor het detecteren en analyseren van gassen via een warmte-koelcyclus.







