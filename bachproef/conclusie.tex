%%=============================================================================
%% Conclusie
%%=============================================================================

\chapter{Conclusie}%
\label{ch:conclusie}

% TODO: Trek een duidelijke conclusie, in de vorm van een antwoord op de
% onderzoeksvra(a)g(en). Wat was jouw bijdrage aan het onderzoeksdomein en
% hoe biedt dit meerwaarde aan het vakgebied/doelgroep? 
% Reflecteer kritisch over het resultaat. In Engelse teksten wordt deze sectie
% ``Discussion'' genoemd. Had je deze uitkomst verwacht? Zijn er zaken die nog
% niet duidelijk zijn?
% Heeft het onderzoek geleid tot nieuwe vragen die uitnodigen tot verder 
%onderzoek?

%centrale onderzoeksvraag: Hoe geschikt zijn goedkope sensoren om gassen in stalomgevingen te meten?
%deelonderzoeksvragen:  - Hoe gaat een MQ-gassensor te werk?
%                       - Hoe kunnen deze gassensoren de luchtsamenstelling meten?
%                       - Zijn deze gassensoren geschikt om de luchtsamenstelling uit te lezen?
%                       - Hoe werkt een warmte-koelcyclus in een sensor en heeft dit voordelen?



In deze studie werd onderzocht hoe geschikt goedkope sensoren zijn om gassen in stalomgeving te meten. De geteste gassensoren zijn de goedkope MQ-sensoren, die kunnen worden bestuurd via een microcontroller zoals Arduino. Er werd onderzocht welke gassen er in een stalomgeving voorkomen en hoe een MQ-sensor exact werkt. Via de informatie uit de datasheets werd berekend hoe deze sensor de luchtsamenstelling kon meten. 
Zo is er tot de conclusie gekomen dat deze MQ-sensoren niet geschikt zijn om de luchtsamenstelling uit te lezen. De MQ-sensoren zijn zeer instabiel en de luchtsamenstelling is een berekende schatting op basis van 1 waarde. 
%TODO Na de R\textsubscript{0}-waarde te berekenen kunnen sommige gassen wel ongeveer correct worden ingeschat, zoals bijvoorbeeld CO\textsubscript{2}, Maar andere gassen worden totaal verkeerd ingeschat. 
Wat deze gassensoren wel kunnen is detecteren of er een te grote hoeveelheid gas aanwezig is, dit kan ze bruikbaar maken als een goedkope monitor die via een alarmsignaal zou kunnen laten weten wanneer de omgeving een te slechte luchtkwaliteit heeft.
Ook bevatten de datasheets niet heel veel informatie waardoor het kalibratieproces moeilijk kan verlopen. 

Hiernaast werd ook de toepassing van een warmte-koelcyclus uitgetest.
%TODO stabieler dan andere?
Het onderscheiden van verschillende gassen via de curve die door deze warmte-koelcyclus wordt gemaakt is iets wat zeker verder onderzocht kan worden. Aangezien elk gas zich op een andere manier gedraagt en TODO






