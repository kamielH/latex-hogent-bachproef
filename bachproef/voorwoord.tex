%%=============================================================================
%% Voorwoord
%%=============================================================================

\chapter*{\IfLanguageName{dutch}{Woord vooraf}{Preface}}%
\label{ch:voorwoord}

%% TODO:
%% Het voorwoord is het enige deel van de bachelorproef waar je vanuit je
%% eigen standpunt (``ik-vorm'') mag schrijven. Je kan hier bv. motiveren
%% waarom jij het onderwerp wil bespreken.
%% Vergeet ook niet te bedanken wie je geholpen/gesteund/... heeft

Deze bachelorproef is geschreven om te voldoen aan de afstudeereisen van de opleiding Toegepaste Informatica aan de Hogeschool Gent. Ik ben van februari tot mei 2024 bezig geweest met het onderzoeken en uitschrijven van mijn bachelorproef.\par

Het onderwerp van de bachelorproef werd me aangereikt via ILVO Vlaanderen. In het specifiek door Pieter-Jan De Temmerman, die ook mijn co-promotor was.\par

ILVO is een afkorting voor het Instituut voor Landbouw-, Visserij- en Voedingsonderzoek. Dit bedrijf heb ik gekozen omdat ik naast IT ook een grote interesse in landbouw heb. Het leek me daarom interessant om mijn bachelorproef in deze richting te schrijven.\par

Het onderwerp van deze proef leek me zeer interessant omdat het een unieke kans was om mijn kennis van informatica om te zetten naar de praktijk. Door gebruik te maken van een Arduino kon ik de goedkope MQ-sensoren lezen en naar een databank versturen. Omdat ik geen voorafgaande kennis had van Arduino, verliepen de eerste weken wat traag. Maar door het noodzakelijke leerproces te respecteren, voel ik me er nu echter comfortabel mee. Ik heb veel plezier gevonden in het schrijven van scripts en het probleemoplossend denken. Zo zal ik zeker in de toekomst zeker nog meer projecten maken met Arduino.\par

Ook wil ik een paar mensen bedanken die hebben meegeholpen met mijn bachelorproef te onderzoeken en te schrijven.
Eerst en vooral wil ik mijn co-promotor Dhr. Pieter-Jan De Temmerman bedanken. Hij heeft mij tot dit onderwerp geïntroduceerd en gaf me de juiste sturing en feedback tijdens het onderzoeksproces.
Daarnaast wil ik mijn promotor Dhr. Johan Van Schoor bedanken voor de nodige feedback en antwoorden op mijn vragen.
Ten slotte wil ik mijn familie en vrienden bedanken omdat zij er voor mij zijn geweest tijdens mijn onderzoeksproces.\par

Ik wens u veel leesplezier toe.\par

Kamiel Halsberghe\par

Gent, 13 mei 2024